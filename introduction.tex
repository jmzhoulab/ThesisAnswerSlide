\IEEEPARstart{I}n the past few decades, complex networks have been researched and applied widely in a variety of fields, such as network control systems, manufacturing systems, economic systems, modeling production system, communication systems and so on. Control and synchronization issues of large-scale networks system have attracted a large number of attention(see \cite{syn4,syn2,syn1,syn3}).
As is well known, the environment of network constantly change over time because of unpredictably external factors, such as alternation of seasons and climate changes. Meanwhile, the relationships among nodes in the networks are also changing with the variation of the environment. To simulate the randomly changing environment, a number of authors apply Markovian chain to portray variable environment of network. The network has different coupling structures in different states of Markovian chain. Synchronization analysis issues on complex networks with Markovian switching has attracted a sea of attention(see \cite{A-nonhomoMarkovian,p10Markovian,p11Markovian,ChenT_event_triggered,A-randomMarkovianDelay,nonlinearly-coupled}). 

In general, the network cannot achieve synchronization by itself, so many control strategies, such as global control, intermittent control, pinning control and so on, are taken into account to force the network to be stable. Among these control strategies, pinning control is most effective and  economical because it is easily realizable by controlling a partial of the nodes instead of all nodes in the network(see \cite{Pin_control_nodeset,pinning1,pinning2,pinning3}). In\cite{pinning-control}, Chen \emph{et al.} used a single controller to research synchronization issue of complex networks. Admittedly, these literatures use pinning control strategy to show a good effect on network synchronization control problem. However, these authors' pinning control strategy basis on continuous information transmission which lead to unnecessary bandwidth and energy consumption. In order to reduce sampling rate, some investigators put forward periodic intermittent controller without compromising stability and performance of the control system(see \cite{Periodically1,Periodically2,Periodically3,sampledata-control}). In particular, by enlarging the interval between the sampling points can further reduce communication. The disadvantage of periodic intermittent control is that the sampling rate is fixed. In previously mentioned works, each node obtains its own state and neighbor states spontaneously or in a fixed sampling rate, which increased communication load and cost much energy.
The ideal sampling procedure is to determine the sampling point according to the system needs. In order to make up for this deficiency, Recently, event-triggered control strategy is proposed and has attracted considerable research attention in the past decades(see \cite{comparison,even_triggered,even_triggered_multi,p2-linear-driven,self-triggered,c4,c5}). {\AA}str\"{o}m and Bernhardsson in \cite{comparison} pointed out that the event-based sampling technique showed better performance than sampling periodically in time for some simple systems. Dimarogonas \emph{et al.} in \cite{even_triggered_multi} considered centralized and decentralized formulation of event-driven strategies for multi-agent systems. Specially, it is practical that the event-based control is utilized for networked systems with limited resources and a number of researchers focus attention on distributed control of networked systems. In \cite{p1-nonlinear-triger}, Guinaldo \emph{et al.} presented distributed event-triggered control for non-reliable networks and obtained expressions for the delay bound and derived the maximum number of consecutive packet losses for different scenarios. Moreover, some investigators proposed coupling formulation based on event-triggered synchronization issue. For instance, in \cite{A-Distributed-event-triggered}, Frazzoli and Dimarogonas \emph{et al.} considered centralized and distributed formulation of event-driven strategies setup by which continuous measuring of the neighbors' states.

According to the consistency of the node triggered instance, event-triggered control strategy can be divided into centralized event triggered strategy (see \cite{eventcede,event-triggered-consensus}) and decentralized event triggered strategy(see \cite{c4,c5,Eventmultiagentswitching,Eventmultiagentnoises,Eventmultiagentdelay,WSNs,self-triggered}). According to the node state continuity of the triggered rule, event-triggered control strategy can be divided into continuous monitoring and discrete monitoring(see \cite{A-linear-event-triggered,ChenT_event_triggered}). In continuous monitoring, triggered rule depends on the current state information of its and neighbors. Yet, in discrete monitoring, triggered rule only depends on the current latest event-triggered instance state information of its and neighbors, which reduce monitoring costs compared with continuous monitoring. To the best our knowledge, few paper in the open literature considers the synchronization problem basis on continuous monitoring and discrete monitoring event-triggered to date except for\cite{A-linear-event-triggered,ChenT_event_triggered}. Authors in\cite{A-linear-event-triggered} assume the topological structure and coupling structure of complex networks are fixed, which is conservative to some extent. Subsequently, Lu \emph{et al.} in \cite{ChenT_event_triggered} employed the event-triggered strategy in both coupling configuration and pinning control terms to realize stability in coupled dynamical systems with Markovian switching couplings and variable pinned node set, but only focus on  linear coupling configuration. As we all know, the information of complex networks between nodes seldom observed directly, but indirect information through nonlinear function may be available. Therefore, nonlinear coupling structure of complex networks has wider application compared with the linear one.

It is generally known, information transmission between the nodes of complex network will block and appear interrupt for the moment. So some investigators considered synchronization of complex networks with randomly occurred coupling(see \cite{AN-discrete-time-stochastic,A-nonlinear-randomly-switching,randomly-control}). In \cite{randomly-control}, Tang \emph{et al.} showed distributed synchronization of coupled neural networks with randomly occurring control by using Bernoulli stochastic variables to describe the occurrences of distributed adaptive control and updating law according to certain probabilities. Besides, coupling  structure is random variation or switching, sometimes coupling strength is variable too. Cao \emph{et al.} in \cite{A-randomMarkovianDelay} investigated synchronization in an array of coupled neural networks with Markovian jumping and random coupling strength.

    Motivated by the above discussion, this paper studies the synchronization problem of complex networks with Markovian switching by using event-triggered control strategy. In order to make the network system to achieve synchronization, a pinning controller with Markovian switching is employed. By constructing a novel stochastic Lyapunov-Krasovskii function, using the properties of Weiner process and some inequalities, the sufficient conditions of synchronization about continuous monitoring and discrete monitoring are derived and proved theoretically. Numerical simulations are finally given to demonstrate the effectiveness of the theoretical results. The main highlights of this paper are listed as follows:
(1) The model involves Markovian switching, and random occurring coupling and stochastic changing coupling strength are described by two random variables. (2) Nonlinearly coupled structure is considered.
(3) Different from the most existing literatures\cite{uncertain-pro2,uncertain-pro5}, both continuous monitoring and discrete monitoring event triggered schemes are developed separately to reduce the information transfer frequently between nodes, which improve the ability of anti-interference of the network.
(4) In numerical simulations part, we give the comparison of four different triggered rules under continuous monitoring and discrete monitoring.

    The rest of this paper is organized as follows. Section \ref{sec2} gives some preliminaries and the problem formulation, meanwhile, some necessary lemmas and assumptions are also given in this section. In sections \ref{sec3}, for continuous and discrete  monitoring scenarios, we present the main results that the nonlinearly and randomly coupled system with Markovian switching is pinned a homogenous preassigned trajectory based on the event-triggering rules of diffusion and pinning terms. Application examples are presented in section \ref{numerical}. Conclusions are finally drawn in section \ref{conclusion}.

    \emph{Notation}: Throughout this paper, if not explicitly stated, matrices are assumed to have compatible dimensions. $0$ denotes the zero matrix. For a vector $x$, $x^\top$ denotes the transpose vector and $\|\cdot\|$ denotes the Euclidean norm of the vector. $A>0$ or $A<0$ denotes that the matrix $A$ is symmetric and positive or negative definite matrix. $E$ stands for the mathematical expectation. The symbol $\otimes$ denotes the Kronecker product. 